% 第一章:示例(中文排版)
\chapter{绪论}

这本学习笔记用于记录地震偏移成像的相关内容,主要包含以下几个方面:
\begin{itemize}
  \item 基本数理知识
  \item 常用的正演算子
  \item 时间域偏移成像方法
  \item 深度域偏移成像方法
  \item 成像条件
  \item 等等 
\end{itemize}

\section{动机}
LaTeX 在学术写作中非常常见,结合 XeLaTeX 与 \verb+ctex+ 宏包可以方便地进行中文类型设置。通过 \verb+book+ 文类管理章节结构,能清晰维护一个长期的讲义项目。关于中文文档支持与配置,可参见 \parencite{ctex-doc,zhang2020example}。

\section{结构约定}
建议将每一章作为一个独立的 `chapters/chap-*.tex` 文件,避免主文件过于臃肿。章节文件只写内容,不需重复导言区设置。